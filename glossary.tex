% acronyms
\newglossaryentry{usb}{
  type=\acronymtype,
  name={USB},
  description={Universal Serial Bus}
}
\newglossaryentry{os}{
  type=\acronymtype,
  name={OS},
  description={operating system}
}
\newglossaryentry{cpu}{
  type=\acronymtype,
  name={CPU},
  description={central processing unit}
}
\newglossaryentry{raspi}{
  type=\acronymtype,
  name={RasPi},
  first={Raspberry Pi (RasPi)},
  description={\gls{raspberrypi}}
}
\newglossaryentry{CD}{
  type=\acronymtype,
  name={CD},
  description={compact disc}
}
\newglossaryentry{vm}{
  type=\acronymtype,
  name={VM},
  plural={VMs},
  first={virtual machine (VM)},
  firstplural={virtual machines (VMs)},
  description={virtual machine}
}
\newglossaryentry{eg}{
  type=\acronymtype,
  name={e.g.},
  description={example given}
}
\newglossaryentry{bash}{
  type=\acronymtype,
  name={Bash},
  description={Bourne-again shell}
}
\newglossaryentry{url}{
  type=\acronymtype,
  name={URL},
  description={uniform resource locator}
}
\newglossaryentry{aur}{
  type=\acronymtype,
  name={AUR},
  first={\gls{arch} User Repository (AUR)},
  description={\gls{arch} User Repository}
}
\newglossaryentry{rc}{
  type=\acronymtype,
  name={RC},
  first={Release Candidate (RC)},
  description={\gls{relcan}}
}
\newglossaryentry{gb}{
  type=\acronymtype,
  name={GB},
  description={Gigabyte}
}
\newglossaryentry{gib}{
  type=\acronymtype,
  name={GiB},
  description={Gibibyte}
}
\newglossaryentry{gui}{
  type=\acronymtype,
  name={GUI},
  description={graphical user interface}
}
\newglossaryentry{ssh}{
  type=\acronymtype,
  name={SSH},
  description={secure shell}
}
\newglossaryentry{cto}{
  type=\acronymtype,
  name={CTO},
  description={chief technical officer}
}
\newglossaryentry{wep}{
  type=\acronymtype,
  name={WEP},
  first={WEP (Wired Equivalent Privacy)},
  description={Wired Equivalent Privacy}
}
\newglossaryentry{wpa}{
  type=\acronymtype,
  name={WPA},
  first={WiFi Protected Access (WPA)},
  description={WiFi Protected Access}
}
\newglossaryentry{wpa2}{
  type=\acronymtype,
  name={WPA2},
  first={WiFi Protected Access 2 (WPA2)},
  description={WiFi Protected Access 2}
}
\newglossaryentry{iv}{
  type=\acronymtype,
  name={IV},
  first={Initialization Vector (IV)},
  description={Initialization Vector}
}
\newglossaryentry{icv}{
  type=\acronymtype,
  name={ICV},
  first={Integrity Check Value (ICV)},
  description={Integrity Check Value}
}
\newglossaryentry{rc4}{
  type=\acronymtype,
  name={RC4},
  description={Rivest Cipher 4}
}
\newglossaryentry{crc}{
  type=\acronymtype,
  name={CRC},
  first={Cyclic Redundancy Check (CRC)},
  description={Cyclic Redundancy Check}
}
\newglossaryentry{crc32}{
  type=\acronymtype,
  name={CRC32},
  description={\gls{crc} with a resulting 32 bit hash value}
}
\newglossaryentry{ksa}{
  type=\acronymtype,
  name={KSA},
  first={Key Scheduling Algorithm (KSA)},
  description={Key Scheduling Algorithm}
}
\newglossaryentry{prga}{
  type=\acronymtype,
  name={PRGA},
  first={Pseudo Random Generation Algorithm (PRGA)},
  plural={PRGAs},
  description={Pseudo Random Generation Algorithm}
}
\newglossaryentry{ap}{
  type=\acronymtype,
  name={AP},
  plural={APs},
  first={access point (AP)},
  description={access point}
}
\newglossaryentry{psk}{
  type=\acronymtype,
  name={PSK},
  first={pre shared key (PSK)},
  description={pre shared key}
}
\newglossaryentry{eap}{
  type=\acronymtype,
  name={EAP},
  first={Extensible Authentication Protocol (EAP)},
  description={Extensible Authentication Protocol}
}
\newglossaryentry{tkip}{
  type=\acronymtype,
  name={TKIP},
  first={Temporal Key Integrity Protocol (TKIP)},
  description={Temporal Key Integrity Protocol}
}
\newglossaryentry{ccmp}{
  type=\acronymtype,
  name={CCMP},
  %first={Counter-Mode \gls{cbc}-\gls{mac} Protocol (CCMP)},
  description={counter-mode cipher block chaining message protocol}
}
\newglossaryentry{cbc}{
  type=\acronymtype,
  name={CBC},
  first={Cipher Block Chaining (CBC)},
  description={Cipher Block Chaining}
}
\newglossaryentry{aes}{
  type=\acronymtype,
  name={AES},
  first={Advanced Encryption Standard (AES)},
  description={Advanced Encryption Standard}
}
\newglossaryentry{radius}{
  type=\acronymtype,
  name={RADIUS},
  first={Remote Authentication Dial-In User Service (RADIUS)},
  description={Remote Authentication Dial-In User Service}
}
\newglossaryentry{hmac}{
  type=\acronymtype,
  name={HMAC},
  first={hash-based message authentication (HMAC)},
  description={hash-based message authentication}
}
\newglossaryentry{ssid}{
  type=\acronymtype,
  name={SSID},
  plural={SSIDs},
  first={service set identifier (SSID)},
  description={Service Set Identifier}
}
\newglossaryentry{wps}{
  type=\acronymtype,
  name={WPS},
  first={WiFi Protected Setup (WPS)},
  description={WiFi Protected Setup}
}
\newglossaryentry{gcc}{
  type=\acronymtype,
  name={GCC},
  first={GNU Compiler Collection (GCC)},
  description={GNU Compiler Collection -- a compiler suite (originally known as GNU C Compiler and back then only used for compiling C) for many different languages like C(++), ObjectiveC, Java and more}
}
\newglossaryentry{macad}{
  type=\acronymtype,
  name={MAC},
  first={Media Access Control (MAC)},
  description={Media Access Control -- a unique identifier assigned to network interfaces for communications on the physical network segment}
}
\newglossaryentry{mac}{
  type=\acronymtype,
  name={MAC},
  first={Message Authentication Code (MAC)},
  description={Message Authentication Code}
}
\newglossaryentry{mic}{
  type=\acronymtype,
  name={MIC},
  first={Message Integrity Check (MIC)},
  description={Message Integrity Check}
}
\newglossaryentry{osi}{
  type=\acronymtype,
  name={OSI},
  first={Open Systems Interconnection (OSI)},
  description={Open Systems Interconnection}
}
\newglossaryentry{nfc}{
  type=\acronymtype,
  name={NFC},
  first={Near Field Communication (NFC)},
  description={Near Field Communication}
}
\newglossaryentry{pin}{
  type=\acronymtype,
  name={PIN},
  first={PIN},
  plural={PINs},
  description={Personal Identification Number}
}
\newglossaryentry{ghz}{
  type=\acronymtype,
  name={GHz},
  description={Gigahertz}
}
\newglossaryentry{pbkdf2}{
  type=\acronymtype,
  name={PBKDF2},
  first={Password-Based Key Derivation Function 2 (PBKDF2)},
  description={Password-Based Key Derivation Function 2}
}
\newglossaryentry{pmk}{
  type=\acronymtype,
  name={PMK},
  first={Preshared Master Key (PMK)},
  description={Preshared Master Key}
}

% glossary -------------------------------------------
\newglossaryentry{kernel}{
  name={Kernel},
  description={q computer program that manages input/output requests from software, and translates them into data processing instructions for the \gls{cpu} and other electronic components of a computer. The kernel is a fundamental part of a modern computer's \gls{os}~\cite{wiki_kernel}}
}
\newglossaryentry{arm}{
  name={ARM},
  description={the ARM-architecture is a design for microprocessors developed 1983 by the company Acorn and maintained by ARM Ltd. since 1990}
}
\newglossaryentry{linux}{
  name={Linux},
  description={a \gls{unix}-like and mostly POSIX-compliant computer \gls{os} assembled under the model of free and open-source software development and distribution. The defining component of Linux is the Linux \gls{kernel}}
}
\newglossaryentry{kali}{
  name={Kali \gls{linux}},
  description={a \gls{debian}-derived \gls{linux} distribution designed for digital forensics and penetration testing. It is maintained and funded by Offensive Security Ltd\footnote{About Kali Linux -- \url{https://www.kali.org/about-us/}}}
}
\newglossaryentry{debian}{
  name={Debian},
  description={a \gls{unix}-like computer \gls{os} and a \gls{linux} distribution that is composed entirely of free and open-source software.\footnote{Debian -- The Universal Operation System -- \url{https://www.debian.org/index.en.html}} The whole project was founded by \gls{murdock}}
}
\newglossaryentry{raspberrypi}{
  name={Raspberry Pi},
  description={a series of credit card–sized single-board computers by the Raspberry Pi Foundation with the intention of promoting the teaching of basic computer science in schools and developing countries\footnote{Raspberry Pi FAQs -- \url{https://www.raspberrypi.org/help/faqs}}}
}
\newglossaryentry{deboot}{
  name={\texttt{debootstrap}},
  description={a tool which allows to install a \gls{debian} base system into a directory of another, already installed system or to create bootable disk images. It doesn't require an installation \gls{cd}, just access to a \gls{debian} repository\footnote{Debootstrap - Debian Wiki -- \url{https://wiki.debian.org/Debootstrap}}}
}
\newglossaryentry{torvalds}{
  name={Linus Torvalds},
  description={creator and long time principal developer of the \gls{linux} \gls{kernel}, creator of \gls{git}}
}
\newglossaryentry{murdock}{
  name={Ian Murdock},
  description={founder of the \gls{debian}-project and \gls{cto} of the Linux Foundation}
}
\newglossaryentry{atowns}{
  name={Anthony Towns},
  description={author of debootstrap}
}
\newglossaryentry{thinclient}{
  name={thin-client},
  plural={thin-clients},
  description={a computer or a program that depends heavily on another computer (its server) to fulfill its computational roles}
}
\newglossaryentry{git}{
  name={Git},
  description={distributed revision control system, created by \gls{torvalds}}
}
\newglossaryentry{unix}{
  name={Unix},
  description={a family of multitasking, multiuser computer \gls{os}}
}
\newglossaryentry{chroot}{
  name={\texttt{chroot}},
  description={an operation that changes the apparent root directory for the current running process and their children. A program that is run in such a modified environment cannot access files and commands outside that environmental directory tree\footnote{Change root - ArchWiki -- \url{https://wiki.archlinux.org/index.php/Change_root}}}
}
\newglossaryentry{tarball}{
  name={tarball},
  description={the name that describes a group or archive of files that are bundled together usually having the \texttt{.tar} file extension}
}
\newglossaryentry{apt}{
  name={\texttt{apt(--get)} / \texttt{aptitude}},
  description={a set of core tools inside \gls{debian} for un-/installing and updating applications}
}
\newglossaryentry{dpkg}{
  name={\texttt{dpkg}},
  description={the software at the base of the package management system in \gls{debian}\footnote{dpkg - Linux man page -- \url{http://linux.die.net/man/1/dpkg}}}
}
\newglossaryentry{grml}{
  name={grml},
  description={a bootable live system based on \gls{debian}. It includes a collection of GNU/\gls{linux} software especially for system administrators and is especially well suited for administrative tasks like installation, deployment and system rescue\footnote{grml.org - Debian Live system / CD for sysadmins and texttool-users -- \url{https://grml.org/}}}
}
\newglossaryentry{knoppix}{
  name={KNOPPIX},
  description={an \gls{os} based on \gls{debian} designed to be run directly from a live \gls{cd}or a USB flash drive, one of the first of its kind for any \gls{os}. It was developed by, and named after, \gls{linux} consultant Klaus Knopper, a german electrical engineer and free software developer}
}
\newglossaryentry{ubuntu}{
  name={Ubuntu},
  description={a \gls{debian} based \gls{linux} \gls{os} and distribution, with Unity as its default desktop environment for personal computers, including smartphones in later versions, and servers\footnote{About Ubuntu -- \url{http://www.ubuntu.com/about/about-ubuntu}}}
}
\newglossaryentry{arch}{
  name={Arch \gls{linux}},
  description={an independently developed, i686/x86-64 general purpose GNU/\gls{linux} distribution versatile enough to suit any role. Development focuses on simplicity, minimalism, and code elegance\footnote{Arch Linux - ArchWiki -- \url{https://wiki.archlinux.org/index.php/Arch_Linux}}}
}
\newglossaryentry{mkdir}{
  name={\texttt{mkdir}},
  description={a command for \gls{unix}-based \gls{os}s to create a directory\footnote{mkdir - Linux man page -- \url{http://linux.die.net/man/1/mkdir}}}
}
\newglossaryentry{cd}{
  name={\texttt{cd}},
  description={a command for \gls{unix}-based \gls{os}s to change the current working directory\footnote{cd - Linux man page -- \url{http://linux.die.net/man/1/cd}}}
}
\newglossaryentry{dd}{
  name={\texttt{dd}},
  description={a tool for \gls{unix}-based \gls{os}s for copying and converting files. It's sometimes referred to as ``disk / data destroyer'' due to the tools capabilities like being able to directly write to a disk regardless of \gls{eg} any present filesystem.\footnote{dd - Linux man page -- \url{http://linux.die.net/man/1/dd}}}
}
\newglossaryentry{relcan}{
  name={release candidate},
  description={version of a program that is nearly ready for release but may still have a few bugs; the status between beta version and release version~\cite{wiki_release_candidate}}
}
\newglossaryentry{pfsense}{
  name={pfSense},
  description={an open source software distribution that is based on \gls{freebsd} with its target usecases as \gls{os} for routers}
}
\newglossaryentry{freebsd}{
  name={FreeBSD},
  description={a free, \gls{unix}-like computer \gls{os}. It has similarities to \gls{linux} while it differs in terms of delivering things like device drivers, the \gls{kernel} and its own userland out of the box that for example \gls{linux} doesn't.\footnote{FreeBSD: the other free UNIX family -- \url{http://www.informit.com/articles/article.aspx?p=439601}}}
}
\newglossaryentry{jail}{
  name={jail},
  description={a separated environment underneath an already existing \gls{os} that acts as a sandbox and can be used to lock processes / users into said subsystem and not giving them access to the main \gls{os} and therefore increasing security}
}
\newglossaryentry{lsblk}{
  name={\texttt{lsblk}},
  description={lists information about all or the specified block devices and therefore needs to read the \texttt{sysfs} filesystem to gather information\footnote{lsblk - Linux man page -- \url{http://linux.die.net/man/8/lsblk}}}
}
\newglossaryentry{qemu}{
  name={\texttt{qemu}},
  description={a generic and open source machine emulator and virtualizer\footnote{QEMU - Wiki -- \url{http://wiki.qemu.org/Main_Page}}}
}
\newglossaryentry{rivest}{
  name={Ron Rivest},
  description={inventor of the \gls{rc4}-algorithm}
}
\newglossaryentry{xor}{
  name={XOR},
  description={logical operation, also known as exclusive disjunction}
}
\newglossaryentry{sbox}{
  name={S-Box},
  description={allows for substitution, used for symmetric key algorithms}
}
\newglossaryentry{osimodel}{
  name={OSI model},
  description={a conceptual model for functions of a telecommunication or computing system without regard to their underlying internal structure and technology}
}
\newglossaryentry{diffhell}{
  name={Diffie-Hellman key exchange},
  description={method of securely exchanging cryptographic keys over a public channel}
}
\newglossaryentry{nonce}{
  name={nonce},
  plural={nonces},
  description={an arbitrary number that may only be used once}
}
\newglossaryentry{ifconfig}{
  name={\texttt{ifconfig}},
  description={allows to configure network interfaces\footnote{ifconfig - Linux man page -- \url{http://linux.die.net/man/8/ifconfig}}}
}
