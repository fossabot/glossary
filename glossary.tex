% acronyms
\newglossaryentry{usb}{
  type=\acronymtype,
  name={USB},
  description={universal serial bus}
}
\newglossaryentry{os}{
  type=\acronymtype,
  name={OS},
  plural={OSs},
  firstplural={operation systems (OSs)},
  description={operating system}
}
\newglossaryentry{cpu}{
  type=\acronymtype,
  name={CPU},
  first={central processing unit (CPU)},
  description={central processing unit}
}
\newglossaryentry{gpu}{
  type=\acronymtype,
  name={GPU},
  first={graphics processing unit (GPU)},
  description={graphics processing uni}
}
\newglossaryentry{raspi}{
  type=\acronymtype,
  name={RasPi},
  first={Raspberry Pi (RasPi)},
  description={\gls{raspberrypi}}
}
\newglossaryentry{CD}{
  type=\acronymtype,
  name={CD},
  description={compact disc}
}
\newglossaryentry{vm}{
  type=\acronymtype,
  name={VM},
  plural={VMs},
  first={virtual machine (VM)},
  firstplural={virtual machines (VMs)},
  description={virtual machine}
}
\newglossaryentry{eg}{
  type=\acronymtype,
  name={e.g.},
  description={example given}
}
\newglossaryentry{bash}{
  type=\acronymtype,
  name={Bash},
  description={Bourne-again shell}
}
\newglossaryentry{url}{
  type=\acronymtype,
  name={URL},
  description={uniform resource locator}
}
\newglossaryentry{aur}{
  type=\acronymtype,
  name={AUR},
  first={\gls{arch} User Repository (AUR)},
  description={\gls{arch} User Repository}
}
\newglossaryentry{rc}{
  type=\acronymtype,
  name={RC},
  first={Release Candidate (RC)},
  description={\gls{relcan}}
}
\newglossaryentry{mb}{
  type=\acronymtype,
  name={MB},
  description={Megabyte}
}
\newglossaryentry{gb}{
  type=\acronymtype,
  name={GB},
  description={Gigabyte}
}
\newglossaryentry{gib}{
  type=\acronymtype,
  name={GiB},
  description={Gibibyte}
}
\newglossaryentry{gui}{
  type=\acronymtype,
  name={GUI},
  description={graphical user interface}
}
\newglossaryentry{ssh}{
  type=\acronymtype,
  name={SSH},
  description={secure shell}
}
\newglossaryentry{cto}{
  type=\acronymtype,
  name={CTO},
  description={chief technical officer}
}
\newglossaryentry{wep}{
  type=\acronymtype,
  name={WEP},
  first={wired equivalent privacy (WEP))},
  description={wired equivalent privacy}
}
\newglossaryentry{wpa}{
  type=\acronymtype,
  name={WPA},
  first={WiFi protected access (WPA)},
  description={WiFi protected access}
}
\newglossaryentry{wpa2}{
  type=\acronymtype,
  name={WPA2},
  first={WiFi protected access 2 (WPA2)},
  description={\gls{wpa}2}
}
\newglossaryentry{iv}{
  type=\acronymtype,
  name={IV},
  first={initialization vector (IV)},
  description={initialization vector}
}
\newglossaryentry{icv}{
  type=\acronymtype,
  name={ICV},
  first={integrity check value (ICV)},
  description={integrity check value}
}
\newglossaryentry{rc4}{
  type=\acronymtype,
  name={RC4},
  description={\gls{rivest} cipher 4}
}
\newglossaryentry{crc}{
  type=\acronymtype,
  name={CRC},
  first={cyclic redundancy check (CRC)},
  description={cyclic redundancy check}
}
\newglossaryentry{crc32}{
  type=\acronymtype,
  name={CRC32},
  description={\gls{crc} with a resulting 32 bit hash value}
}
\newglossaryentry{ksa}{
  type=\acronymtype,
  name={KSA},
  first={key scheduling algorithm (KSA)},
  description={key scheduling algorithm}
}
\newglossaryentry{prga}{
  type=\acronymtype,
  name={PRGA},
  first={pseudo random generation algorithm (PRGA)},
  plural={PRGAs},
  description={pseudo random generation algorithm}
}
\newglossaryentry{ap}{
  type=\acronymtype,
  name={AP},
  plural={APs},
  first={access point (AP)},
  description={access point}
}
\newglossaryentry{psk}{
  type=\acronymtype,
  name={PSK},
  first={pre shared key (PSK)},
  description={pre shared key}
}
\newglossaryentry{eap}{
  type=\acronymtype,
  name={EAP},
  first={extensible authentication protocol (EAP)},
  description={extensible authentication protocol}
}
\newglossaryentry{tkip}{
  type=\acronymtype,
  name={TKIP},
  first={temporal key integrity protocol (TKIP)},
  description={temporal key integrity protocol}
}
\newglossaryentry{ccmp}{
  type=\acronymtype,
  name={CCMP},
  %first={Counter-Mode \gls{cbc}-\gls{mac} Protocol (CCMP)},
  description={counter-mode cipher block chaining message protocol}
}
\newglossaryentry{cbc}{
  type=\acronymtype,
  name={CBC},
  first={cipher block chaining (CBC)},
  description={cipher block chaining}
}
\newglossaryentry{aes}{
  type=\acronymtype,
  name={AES},
  first={advanced encryption standard (AES)},
  description={advanced encryption standard}
}
\newglossaryentry{radius}{
  type=\acronymtype,
  name={RADIUS},
  first={remote authentication dial-in user service (RADIUS)},
  description={remote authentication dial-in user service}
}
\newglossaryentry{hmac}{
  type=\acronymtype,
  name={HMAC},
  first={hash-based message authentication (HMAC)},
  description={hash-based message authentication}
}
\newglossaryentry{ssid}{
  type=\acronymtype,
  name={SSID},
  plural={SSIDs},
  first={service set identifier (SSID)},
  description={service set identifier}
}
\newglossaryentry{wps}{
  type=\acronymtype,
  name={WPS},
  first={WiFi protected setup (WPS)},
  description={WiFi protected setup}
}
\newglossaryentry{gcc}{
  type=\acronymtype,
  name={GCC},
  first={GNU compiler collection (GCC)},
  description={\gls{gnu} Compiler Collection -- a compiler suite (originally known as \gls{gnu} C Compiler and back then only used for compiling C) for many different languages like C(++), ObjectiveC, Java and more}
}
\newglossaryentry{macad}{
  type=\acronymtype,
  name={MAC},
  first={media access control (MAC)},
  description={media access control -- a unique identifier assigned to network interfaces for communications on the physical network segment}
}
\newglossaryentry{mac}{
  type=\acronymtype,
  name={MAC},
  first={message authentication code (MAC)},
  description={message authentication code}
}
\newglossaryentry{mic}{
  type=\acronymtype,
  name={MIC},
  first={message integrity check (MIC)},
  description={message integrity check}
}
\newglossaryentry{osi}{
  type=\acronymtype,
  name={OSI},
  first={open systems interconnection (OSI)},
  description={open systems interconnection}
}
\newglossaryentry{nfc}{
  type=\acronymtype,
  name={NFC},
  first={near field communication (NFC)},
  description={near field communication}
}
\newglossaryentry{pin}{
  type=\acronymtype,
  name={PIN},
  first={PIN},
  plural={PINs},
  description={personal identification number}
}
\newglossaryentry{mhz}{
  type=\acronymtype,
  name={MHz},
  description={Megahertz}
}
\newglossaryentry{ghz}{
  type=\acronymtype,
  name={GHz},
  description={Gigahertz}
}
\newglossaryentry{pbkdf2}{
  type=\acronymtype,
  name={PBKDF2},
  first={password-based key derivation function 2 (PBKDF2)},
  description={password-based key derivation function 2}
}
\newglossaryentry{pmk}{
  type=\acronymtype,
  name={PMK},
  first={preshared master key (PMK)},
  description={preshared master key}
}
\newglossaryentry{evm}{
  type=\acronymtype,
  name={EVM},
  first={Ethereum virtual machine (EVM)},
  description={\gls{eth} virtual machine}
}
\newglossaryentry{dht}{
  type=\acronymtype,
  name={DHT},
  plural={DHTs},
  first={distributed hashtable (DHT)},
  firstplural={distributed hashtables (DHTs)},
  description={distributed hashtable}
}
\newglossaryentry{ecdsa}{
  type=\acronymtype,
  name={ECDSA},
  first={elliptic curve digital signature algorithm (ECDSA)},
  description={\gls{ec}\gls{dsa}}
}
\newglossaryentry{ec}{
  type=\acronymtype,
  name={EC},
  first={elliptic curve (EC)},
  description={elliptic-curve [cryptography]}
}
\newglossaryentry{dsa}{
  type=\acronymtype,
  name={DSA},
  first={digital signature algorithm (DSA)},
  description={digital signature algorithm}
}
\newglossaryentry{foss}{
  type=\acronymtype,
  name={FOSS},
  first={free, open-source software (FOSS)},
  description={free, open-source software}
}
\newglossaryentry{gnu}{
  type=\acronymtype,
  name={GNU},
  description={"\gls{gnu_long}'s not \gls{unix}"}
}
\newglossaryentry{api}{
  type=\acronymtype,
  name={API},
  first={application programming interface (API)},
  description={application programming interface}
}
\newglossaryentry{iot}{
  type=\acronymtype,
  name={IoT},
  first={Internet of things (IoT)},
  description={Internet of things}
}
\newglossaryentry{m2m}{
  type=\acronymtype,
  name={M2M},
  first={machine to machine (M2M)},
  description={machine to machine}
}
\newglossaryentry{adb}{
  type=\acronymtype,
  name={ADB},
  first={Android debugging bridge (ADB)},
  description={\gls{android} debugging bridge}
}
\newglossaryentry{sdk}{
  type=\acronymtype,
  name={SDK},
  first={software development kit (SDK)},
  description={software development kit}
}
\newglossaryentry{ndk}{
  type=\acronymtype,
  name={ndk},
  first={native development kit (NDK)},
  description={native development kit}
}
\newglossaryentry{hdmi}{
  type=\acronymtype,
  name={HDMI},
  first={HDMI},
  description={high-definition multimedia interface}
}
\newglossaryentry{ble}{
  type=\acronymtype,
  name={BLE},
  first={Bluetooth low energy (BLE)},
  description={Bluetooth low energy}
}
\newglossaryentry{rfid}{
  type=\acronymtype,
  name={RFID},
  first={radio frequency identification (RFID)},
  description={radio frequency identification}
}
\newglossaryentry{spi}{
  type=\acronymtype,
  name={SPI},
  first={serial peripheral interface (SPI)},
  description={serial peripheral interface}
}
\newglossaryentry{uart}{
  type=\acronymtype,
  name={UART},
  first={universal asynchronous receiver-transmitter (UART)},
  description={universal asynchronous receiver-transmitter}
}
\newglossaryentry{gpio}{
  type=\acronymtype,
  name={GPIO},
  first={general purpose \gls{io} (GPIO)},
  description={general purpose \gls{io}}
}
\newglossaryentry{pwm}{
  type=\acronymtype,
  name={PWM},
  first={pulse width modulation (PWM)},
  description={pulse width modulation}
}
\newglossaryentry{ota}{
  type=\acronymtype,
  name={OTA},
  first={over the air (OTA)},
  description={over the air}
}
\newglossaryentry{soc}{
  type=\acronymtype,
  name={SoC},
  plural={SoCs},
  first={system on a chip (SoC)},
  description={system on a chip}
}
\newglossaryentry{io}{
  type=\acronymtype,
  name={I/O},
  description={input / output}
}
\newglossaryentry{art}{
  type=\acronymtype,
  name={ART},
  first={\gls{android} runtime (ART)},
  description={\gls{android} runtime}
}
\newglossaryentry{hal}{
  type=\acronymtype,
  name={HAL},
  first={hardware abstraction layer (HAL)},
  description={hardware abstraction layer}
}
\newglossaryentry{mmu}{
  type=\acronymtype,
  name={MMU},
  first={memory management unit (MMU)},
  description={memory management unit}
}
\newglossaryentry{ram}{
  type=\acronymtype,
  name={RAM},
  description={random access memory}
}
\newglossaryentry{4p4c}{
  type=\acronymtype,
  name={4P4C},
  description={4 position 4 contact, often incorrectly referred to as \gls{rj}9, \gls{rj}10 or \gls{rj}22 -- see \gls{mod_con}}
}
\newglossaryentry{6p2c}{
  type=\acronymtype,
  name={6P2C},
  description={6 position 2 contact, often incorrectly referred to as \gls{rj}11 -- see \gls{mod_con}}
}
\newglossaryentry{6p4c}{
  type=\acronymtype,
  name={6P4C},
  description={6 position 4 contact, often incorrectly referred to as \gls{rj}14 -- see \gls{mod_con}}
}
\newglossaryentry{6p6c}{
  type=\acronymtype,
  name={6P6C},
  description={6 position 6 contact, often incorrectly referred to as \gls{rj}25 -- see \gls{mod_con}}
}
\newglossaryentry{8p8c}{
  type=\acronymtype,
  name={8P8C},
  description={8 position 8 contact, often incorrectly referred to as \gls{rj}45 -- see \gls{mod_con}}
}
\newglossaryentry{rj}{
  type=\acronymtype,
  name={RJ},
  description={registered jack -- see \gls{mod_con}}
}


% ------------------------------------------------------------------------------
% spacing
% ...
% more spacing
% ...
% and even more spacing
% ...
% glossary ---------------------------------------------------------------------

\newglossaryentry{gnu_long}{
  name={GNU},
  description={the \gls{gnu} \gls{os} is a complete free software system, upward-compatible with \gls{unix}.}
}
\newglossaryentry{kernel}{
  name={Kernel},
  description={a computer program that manages input/output requests from software, and translates them into data processing instructions for the \gls{cpu} and other electronic components of a computer. It is a fundamental part of a modern computer's \gls{os}}
}
\newglossaryentry{arm}{
  name={ARM},
  description={the ARM-architecture is a design for microprocessors developed 1983 by the company Acorn and maintained by ARM Ltd. since 1990}
}
\newglossaryentry{posix}{
  name={POSIX},
  description={defines a standard operating system interface and environment, including a command interpreter, and common utility programs to support applications portability at the source code level. It is intended to be used by both application developers and system implementors.\footnote{The Open Group Base Specifications -- \url{http://pubs.opengroup.org/onlinepubs/9699919799}}}
}
\newglossaryentry{linux}{
  name={Linux},
  description={a \gls{unix}-like and mostly \gls{posix}-compliant \gls{os} assembled under the model of \gls{foss} development and distribution. The defining component is the \gls{kernel}}
}
\newglossaryentry{kali}{
  name={Kali \gls{linux}},
  description={a \gls{debian}-derived \gls{linux} distribution designed for digital forensics and penetration testing. It is maintained and funded by Offensive Security Ltd\footnote{About Kali Linux -- \url{https://www.kali.org/about-us}}}
}
\newglossaryentry{debian}{
  name={Debian},
  description={a \gls{unix}-like computer \gls{os} and a \gls{linux} distribution that is composed entirely of free and open-source software.\footnote{Debian -- The Universal Operation System -- \url{https://www.debian.org/index.en.html}}}%The whole project was founded by \gls{murdock}}
}
\newglossaryentry{raspberrypi}{
  name={Raspberry Pi},
  description={a series of credit card sized single-board computers by the Raspberry Pi Foundation with the intention of promoting the teaching of basic computer science in schools and developing countries\footnote{Raspberry Pi FAQs -- \url{https://www.raspberrypi.org/help/faqs}}}
}
\newglossaryentry{deboot}{
  name={\texttt{debootstrap}},
  description={a tool which allows to install a \gls{debian} base system into a directory of another, already installed system or to create bootable disk images. It doesn't require an installation \gls{cd}, just access to a \gls{debian} repository\footnote{Debootstrap - Debian Wiki -- \url{https://wiki.debian.org/Debootstrap}}}
}
\newglossaryentry{torvalds}{
  name={Linus Torvalds},
  description={creator and long time principal developer of the \gls{linux} \gls{kernel}, creator of \gls{git}}
}
\newglossaryentry{murdock}{
  name={Ian Murdock},
  description={founder of the \gls{debian}-project and \gls{cto} of the Linux Foundation}
}
\newglossaryentry{atowns}{
  name={Anthony Towns},
  description={author of debootstrap}
}
\newglossaryentry{thinclient}{
  name={thin-client},
  plural={thin-clients},
  description={a computer or a program that depends heavily on another computer (its server) to fulfill its computational roles}
}
\newglossaryentry{git}{
  name={Git},
  description={distributed revision control system, created by \gls{torvalds}}
}
\newglossaryentry{unix}{
  name={Unix},
  description={a family of multitasking, multiuser computer \glspl{os}}
}
\newglossaryentry{chroot}{
  name={\texttt{chroot}},
  description={an operation that changes the apparent root directory for the current running process and their children. A program that is run in such a modified environment cannot access files and commands outside that environmental directory tree\footnote{Change root - ArchWiki -- \url{https://wiki.archlinux.org/index.php/Change_root}}}
}
\newglossaryentry{tarball}{
  name={tarball},
  description={the name that describes a group or archive of files that are bundled together usually having the \texttt{.tar} file extension}
}
\newglossaryentry{apt}{
  name={\texttt{apt(--get)} / \texttt{aptitude}},
  description={a set of core tools inside \gls{debian} for un-/installing and updating applications}
}
\newglossaryentry{dpkg}{
  name={\texttt{dpkg}},
  description={the software at the base of the package management system in \gls{debian}\footnote{dpkg - Linux man page -- \url{http://linux.die.net/man/1/dpkg}}}
}
\newglossaryentry{grml}{
  name={grml},
  description={a bootable live system based on \gls{debian}. It includes a collection of \gls{gnu}/\gls{linux} software especially for system administrators and is especially well suited for administrative tasks like installation, deployment and system rescue\footnote{grml.org - Debian Live system / CD for sysadmins and texttool-users -- \url{https://grml.org/}}}
}
\newglossaryentry{knoppix}{
  name={KNOPPIX},
  description={an \gls{os} based on \gls{debian} designed to be run directly from a live \gls{cd}or a USB flash drive, one of the first of its kind for any \gls{os}. It was developed by, and named after, \gls{linux} consultant Klaus Knopper, a German electrical engineer and free software developer}
}
\newglossaryentry{ubuntu}{
  name={Ubuntu},
  description={a \gls{debian} based \gls{linux} \gls{os} and distribution, for personal computers, including smartphones in later versions, and servers\footnote{About Ubuntu -- \url{http://www.ubuntu.com/about/about-ubuntu}}}
}
\newglossaryentry{arch}{
  name={Arch \gls{linux}},
  description={an independently developed, i686/x86-64 general purpose \gls{gnu}/\gls{linux} distribution versatile enough to suit any role. Development focuses on simplicity, minimalism, and code elegance\footnote{Arch Linux - ArchWiki -- \url{https://wiki.archlinux.org/index.php/Arch_Linux}}}
}
\newglossaryentry{mkdir}{
  name={\texttt{mkdir}},
  description={a command for \gls{unix}-based \glspl{os} to create a directory\footnote{mkdir - Linux man page -- \url{http://linux.die.net/man/1/mkdir}}}
}
\newglossaryentry{cd}{
  name={\texttt{cd}},
  description={a command for \gls{unix}-based \glspl{os} to change the current working directory\footnote{cd - Linux man page -- \url{http://linux.die.net/man/1/cd}}}
}
\newglossaryentry{dd}{
  name={\texttt{dd}},
  description={a tool for \gls{unix}-based \glspl{os} for copying and converting files. It's sometimes referred to as ``disk / data destroyer'' due to the tools capabilities like being able to directly write to a disk regardless of \gls{eg} any present filesystem.\footnote{dd - Linux man page -- \url{http://linux.die.net/man/1/dd}}}
}
\newglossaryentry{wget}{
  name={\texttt{wget}},
  description={a command for \gls{unix}-based \glspl{os} for non-interactive download of files from the web supporting HTTP, HTTPS and FTP} % TODO entries for HTTP[S] & FTP
}
\newglossaryentry{unzip}{
  name={\texttt{unzip}},
  description={a command for \gls{unix}-based \glspl{os} to list, test and extract compressed files in a ZIP archive}
}
\newglossaryentry{relcan}{
  name={release candidate},
  description={version of a program that is nearly ready for release but may still have a few bugs; the status between beta version and release version~\cite{wiki_release_candidate}}
}
\newglossaryentry{pfsense}{
  name={pfSense},
  description={an open source software distribution that is based on \gls{freebsd} with its target usecases as \gls{os} for routers}
}
\newglossaryentry{freebsd}{
  name={FreeBSD},
  description={a free, \gls{unix}-like computer \gls{os}. It has similarities to \gls{linux} while it differs in terms of delivering things like device drivers, the \gls{kernel} and its own userland out of the box that for example \gls{linux} doesn't.\footnote{FreeBSD: the other free UNIX family -- \url{http://www.informit.com/articles/article.aspx?p=439601}}}
}
\newglossaryentry{jail}{
  name={jail},
  description={a separated environment underneath an already existing \gls{os} that acts as a sandbox and can be used to lock processes / users into said subsystem and not giving them access to the main \gls{os} and therefore increasing security}
}
\newglossaryentry{lsblk}{
  name={\texttt{lsblk}},
  description={lists information about all or the specified block devices and therefore needs to read the \texttt{sysfs} filesystem to gather information\footnote{lsblk - Linux man page -- \url{http://linux.die.net/man/8/lsblk}}}
}
\newglossaryentry{qemu}{
  name={\texttt{qemu}},
  description={a generic and open source machine emulator and virtualizer\footnote{QEMU - Wiki -- \url{http://wiki.qemu.org/Main_Page}}}
}
\newglossaryentry{rivest}{
  name={Ron Rivest},
  description={inventor of the \gls{rc4}-algorithm}
}
\newglossaryentry{xor}{
  name={XOR},
  description={logical operation, also known as exclusive disjunction}
}
\newglossaryentry{sbox}{
  name={S-Box},
  description={allows for substitution, used for symmetric key algorithms}
}
\newglossaryentry{osimodel}{
  name={\gls{osi} model},
  description={a conceptual model for functions of a telecommunication or computing system without regard to their underlying internal structure and technology}
}
\newglossaryentry{diffhell}{
  name={Diffie-Hellman key exchange},
  description={method of securely exchanging cryptographic keys over a public channel}
}
\newglossaryentry{nonce}{
  name={nonce},
  plural={nonces},
  description={an arbitrary number that may only be used once}
}
\newglossaryentry{ifconfig}{
  name={\texttt{ifconfig}},
  description={allows to configure network interfaces\footnote{ifconfig - Linux man page -- \url{http://linux.die.net/man/8/ifconfig}}}
}
\newglossaryentry{eth}{
  name={Ethereum},
  description={a decentralized platform that runs smart contracts: applications that run exactly as programmed without any possibility of downtime, censorship, fraud or third party interference.\footnote{Ethereum -- \url{https://ethereum.org}}}
}
\newglossaryentry{btc}{
  name={Bitcoin},
  description={a worldwide cryptocurrency and digital payment system called the first decentralized digital currency, as the system works without a central repository or single administrator.\footnote{Bitcoin -- \url{https://www.bitcoin.com}}}
}
\newglossaryentry{android}{
  name={Android},
  description={a mobile \gls{os} developed by Google, based on a modified version of the \gls{linux} \gls{kernel} and other \gls{foss} and designed primarily for touchscreen mobile devices such as smartphones and tablets.\footnote{Android -- \url{https://www.android.com}}}
}
\newglossaryentry{i2c}{
  name={I$^2$C},
  description={a multi-master, multi-slave, packet switched, single-ended, serial computer bus}
}
\newglossaryentry{mod_con}{
  name={modular connector},
  description={an electrical connector that was originally designed for use in telephone wiring, but has since been used for many other purposes where the probably most well known applications are for telephone and Ethernet. While often falsely being referred to as registered jacks, the \gls{rj} specifications define the wiring patterns of the jacks, not the physical dimensions or geometry of the connectors, which are actually specified under ISO standard 8877~\cite{iso8877, modular_connector}}
}
